\section{Unique Game Conjecture}

\subsection{Unique Label Cover}

    It is an optimization problem.

    We are given a \textit{bipartite graph} $G((V,W), E)$ (see~\nameref{subsubsec:bipartitegraph}), and given a set of labels $[m] = \{1, 2, \dots, m\}$, and $\forall e = \{ v,w \} \in E$ a permutation $\pi_{vw} : [m] \rightarrow [m]$.

    Find an assignment of labels to the vertices, $\phi : V \cup W \rightarrow [m]$, s.t.~the set $S$ of satisfied edges is as large as possible.
    \[ S =\big{ \{ v,w \} \in E \st v \in V \wedge w \in W \wedge \pi_{vw}(\phi(v)) = \phi(w) \big}   \]

\subsection{Unique Games Conjecture}
    Formulated by Khot in 2002.

    For each $\varepsilon > 0$, it is NP-Hard to determine if, in a Unique Label Cover problem, with $m = m(\varepsilon)$ labels, if:
    \begin{itemize}
        \item Each assignment of labels (each solution) satisfies no more than $\varepsilon \cardinality{E}$ edges, or
        \item There exists an assignment (a solution) which satisfies at least $(1-\varepsilon) \cardinality{E}$ edges
    \end{itemize}

    From this conjecture, came a bunch of theorems stating that, if the UGC holds, than some problem is NP-Hard to approximate better than a specific value. Let us see a couple of examples.

    \begin{theorem}
        If the UGC holds, then $\forall \varepsilon > 0$, it is NP-Hard to approximate Max-Cut to $\alpha_{GW} - \varepsilon$.
    \end{theorem}

    \begin{theorem}
        If the UGC holds, then $\forall \varepsilon > 0$, is it NP-Hard to approximate Vertex-Cover to $2 - \varepsilon$.
    \end{theorem}


\subsection{Constraint Satisfaction Problems (CSPs)}
    CSPs of a constant arity, on a constant-sized alphabet, can be optimally approximated via a SDP-based algorithm in polytime, if the UGC holds.


\subsection{Sparsest Cut problem}
    Sparsest Cut is not a CSP.\@

    Given an undirected graph $G(V,E)$, find a subset $\emptyset \subset S \subset V$ that minimizes
    \[ \Psi(S) = \dfrac{\cardinality{Cut(S, V \setminus S)}}{\cardinality{S} \cdot \cardinality{V \setminus S}} \]

    Let us consider the Sparsest Cut problem for \textbf{Erd\H{o}s-Renyi random graphs}.
    To sample an E-R r.g., one can use the algorithm \ref{alg:er_rg}.

    \input{algorithms/ER_RG.tex}

    %$\expected[\cardinality{Cut(S, V \setminus S)}] =
    %\sum_{i \in S} \sum_{j \in V \setminus S} \prob[\{ i,j \} \in E] =
    %\sum_{i \in S} \sum_{j \in V \setminus S} p =
    %\cardinality{S} \cdot \cardinality{V \setminus S} \cdot p$

    \begin{equation*}
        \begin{split}
            \expected[\cardinality{Cut(S, V \setminus S)}] &=\\
                &= \sum_{i \in S} \sum_{j \in V \setminus S} \prob[\{ i,j \} \in E] =\\
                &= \sum_{i \in S} \sum_{j \in V \setminus S} p = \cardinality{S} \cdot \cardinality{V \setminus S} \cdot p
        \end{split}
    \end{equation*}

    Thus, $\expected_{G \thicksim G(n,p)}[\Psi(S)] =
    \dfrac{\expected[\cardinality{Cut(S, V \setminus S)}]}{\cardinality{S} \cdot \cardinality{V \setminus S}} =
    \dfrac{ \cardinality{S} \cdot \cardinality{V \setminus S} \cdot p}{\cardinality{S} \cdot \cardinality{V \setminus S}} = 
    p$

    We have concluded that
    \[ 0 \leq \cardinality{Cut(S, V \setminus S)} \leq \cardinality{S} \cdot \cardinality{V \setminus S} \]

    This implies that
    \[ \Psi(S) \in [0,1] \]

    Observe that, if $\Psi(S) = 0 \rightarrow$ $S$ is disconnected from $V \setminus S$.

    Our \textit{objective function} is
    \[ \Psi^* = \min_{\emptyset \subset S \subset V} \Psi(S) \]

    \textbf{Leighton-Rao} algorithm for Sparsest-Cut:
    \begin{itemize}
        \item Based on an LP
        \item Geometry plays a pivotal role in its analysis
        \item The algorithm is, essentially, a series of geometric reductions
    \end{itemize}