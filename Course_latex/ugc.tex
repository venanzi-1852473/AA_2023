\section{Unique Game Conjecture}

\subsection{Unique Label Cover}

    It is an optimization problem.

    We are given a \textit{bipartite graph} $G((V,W), E)$ (see~\nameref{subsubsec:bipartitegraph}), and given a set of labels $[m] = \{1, 2, \dots, m\}$, and $\forall e = \{ v,w \} \in E$ a permutation $\pi_{vw} : [m] \rightarrow [m]$.

    Find an assignment of labels to the vertices, $\phi : V \cup W \rightarrow [m]$, s.t.~the set $S$ of satisfied edges is as large as possible.
    \[ S = \{ v,w \} \in E \st v \in V \wedge w \in W \wedge \pi_{vw}(\phi(v)) = \phi(w) \]

\subsection{Unique Games Conjecture}
    Formulated by Khot in 2002.

    For each $\varepsilon > 0$, it is NP-Hard to determine if, in a Unique Label Cover problem, with $m = m(\varepsilon)$ labels, if:
    \begin{itemize}
        \item Each assignment of labels (each solution) satisfies no more than $\varepsilon \cardinality{E}$ edges, or
        \item There exists an assignment (a solution) which satisfies at least $(1-\varepsilon) \cardinality{E}$ edges
    \end{itemize}

    From this conjecture, came a bunch of theorems stating that, if the UGC holds, than some problem is NP-Hard to approximate better than a specific value. Let us see a couple of examples.

    \begin{theorem}
        If the UGC holds, then $\forall \varepsilon > 0$, it is NP-Hard to approximate Max-Cut to $\alpha_{GW} - \varepsilon$.
    \end{theorem}

    \begin{theorem}
        If the UGC holds, then $\forall \varepsilon > 0$, is it NP-Hard to approximate Vertex-Cover to $2 - \varepsilon$.
    \end{theorem}


\subsection{Constraint Satisfaction Problems (CSPs)}
    CSPs of a constant arity, on a constant-sized alphabet, can be optimally approximated via a SDP-based algorithm in polytime, if the UGC holds.


\subsection{Sparsest Cut problem}
    Sparsest Cut is not a CSP.\@

    Given an undirected graph $G(V,E)$, find a subset $\emptyset \subset S \subset V$ that minimizes
    \[ \Psi(S) = \dfrac{\cardinality{Cut(S, V \setminus S)}}{\cardinality{S} \cdot \cardinality{V \setminus S}} \]

    Let us consider the Sparsest Cut problem for \textbf{Erd\H{o}s-Renyi random graphs}.
    To sample an E-R r.g., one can use the algorithm \ref{alg:er_rg}.

    \begin{algorithm}
    \caption{E-R random graph $G(n,p)$}\label{alg:er_rg}
    \begin{algorithmic}%[1]
        \State$S \gets [n]$\@
        \State$E \gets \emptyset$\@
        \For{each $\{i,j\} \in \binom{V}{2}$ }
            \State flip an iid coin with head probability $p$\@
            \If{coin is heads}
                \State $E \gets E \cup \{i,j\}$
            \EndIf\@
        \EndFor\@

        \State\Return~$G(V,E)$
    \end{algorithmic}
\end{algorithm}

    $\expected[\cardinality{Cut(S, V \setminus S)}] =
    \sum_{i \in S} \sum_{j \in V \setminus S} \prob[\{ i,j \} \in E] =
    \sum_{i \in S} \sum_{j \in V \setminus S} p =
    \cardinality{S} \cdot \cardinality{V \setminus S} \cdot p$

    Thus, $\expected_{G \thicksim G(n,p)}[\Psi(S)] =
    \dfrac{\expected[\cardinality{Cut(S, V \setminus S)}]}{\cardinality{S} \cdot \cardinality{V \setminus S}} =
    \dfrac{ \cardinality{S} \cdot \cardinality{V \setminus S} \cdot p}{\cardinality{S} \cdot \cardinality{V \setminus S}} = 
    p$

    We have concluded that
    \[ 0 \leq \cardinality{Cut(S, V \setminus S)} \leq \cardinality{S} \cdot \cardinality{V \setminus S} \]

    This implies that
    \[ \Psi(S) \in [0,1] \]

    Observe that, if $\Psi(S) = 0 \rightarrow$ $S$ is disconnected from $V \setminus S$.

    Our \textit{objective function} is
    \[ \Psi^* = \min_{\emptyset \subset S \subset V} \Psi(S) \]

    \textbf{Leighton-Rao} algorithm for Sparset-Cut:
    \begin{itemize}
        \item Based on an LP
        \item Geometry plays a pivotal role in its analysis
        \item The algorithm is, essentially, a series of geometric reductions
    \end{itemize}


\subsection{Metrics}

    \begin{definition}[Metric]
        A metric $d$ on a set $S$ is a function $d : S \times S \rightarrow \mathbb{R}$, s.t.:
        \begin{enumerate}
            \item $d(X,Y) = d(Y,X)$, $\forall X,Y \in S$
            \item $d(X,X) = 0$, $\forall X \in S$
            \item $d(X,Y) + d(Y,Z) \geq d(X,Z)$, $\forall X,Y,Z \in S$ (Triangle inequality)
        \end{enumerate}
    \end{definition}

    \begin{definition}[Elemetary-cut metric]
        The elementary-cut metric $d_T : S \times S \rightarrow \mathbb{R}$, for $T \subseteq S$, is defined as
        \begin{equation}
            d_T(X,Y) = 
            \begin{cases}
                1 & \text{if } \cardinality{\{X,Y\} \in T } = 1\\
                0 & \text{o/w}
            \end{cases}
        \end{equation}
    \end{definition}

    \begin{lemma}
        If $d_T : S \times S \rightarrow \mathbb{R}$ is an elementary-cut metric, then $d_T$ is a metric.
    \end{lemma}

    \begin{proof}
        Axioms $1.$ and $2.$ are trivial to verify.

        Let $X,Y,Z \in S$:
        \begin{itemize}
            \item if $X,Y,Z \in T$, then $0 = d_T(X,Y) + d_T(Y,Z) \geq d_T(X,Z) = 0$
            \item if $X,Y,Z \not\in T$, then $0 = d_T(X,Y) + d_T(Y,Z) \geq d_T(X,Z) = 0$
        \end{itemize}

        It remains a third case, itself subdivided in three other subcases.

        Assume $\exists A \in \{ T, S \setminus T \}$ s.t.~exactly one of $X,Y,Z$ is in $A$; i.e.~we divide $X,Y$ and $Z$ in the two partitions of $S$.
        \begin{itemize}
            \item if $X \in A$, then $1 = d_T(X,Y) + d_T(Y,Z) \geq d_T(X,Z) = 1$
            \item if $Z \in A$, then $1 = d_T(X,Y) + d_T(Y,Z) \geq d_T(X,Z) = 1$
            \item if $Y \in A$, then $2 = d_T(X,Y) + d_T(Y,Z) \geq d_T(X,Z) = 0$
        \end{itemize}
    \end{proof}

    We want to argue that the Sparsest Cut problem is an optimization problem, over elementary-cut metrics.
    That is, the goal is to minimize $\phi(d_T)$, where $d_T : V \times V \rightarrow \mathbb{R}$ is an elementary-cut metric over $V$, and
    \[ \phi(d_T) = \dfrac{\sum_{\{x,y\} \in E} d_T(X,Y)}{\sum_{\{x,y\} \in \binom{V}{2}} d_T(X,Y)} \]

    \begin{lemma}\label{lemma:metric2}
        If $d_T$ is an elementary-cut metric over $V$, then
        \[ \phi(d_T) = \dfrac{\cardinality{Cut(S, V \setminus S)}}{\cardinality{T} \cdot \cardinality{V \setminus T}} \]
    \end{lemma}

    Let $P$ be any predicate. $[P]$ has value $1$ is $P$ is true, and $0$ if $P$ is false.

    \begin{proof}
        $\phi(d_T) =
        \dfrac{\sum_{\{x,y\} \in E} d_T(X,Y)}{\sum_{\{x,y\} \in \binom{V}{2}} d_T(X,Y)} = 
        \dfrac{\sum_{\{x,y\} \in E} [X \in T \text{ xor } Y \in T]}{\sum_{\{x,y\} \in \binom{V}{2}} [X \in T \text{ xor } Y \in T]} = $

        $\dfrac{\cardinality{Cut(T, V \setminus T)}}{\cardinality{T} \cdot \cardinality{V \setminus T}}$
    \end{proof}

    \begin{definition}[Cut metric]
        Let $d_{T_1}, d_{T_2}, \dots, d_{T_k} : V \times V \rightarrow \mathbb{R}$ be elementary cut metric, and let $\lambda_1, \lambda_2, \dots, \lambda_k > 0$, then $d : V \times V \rightarrow \mathbb{R}$ defined as
        \[ d(X,Y) = \sum_{i=1}^{k}(\lambda_i \cdot d_{T_i}(X,Y)) \]
        is a cut metric.
    \end{definition}

    Cut metric are less "integral" (more "fractional"/flexible) than elementary cut metrics.

    \begin{lemma}\label{lemma:metric3}
        Let $d$ be a cut metric with cuts $T_1, T_2, \dots, T_k$, then
        \[ \phi(d) \geq \min_{i \in [k]} \phi(d_i) \]
    \end{lemma}

    Then, minimizing over cut metrics is equivalent to minimizing over elementary-cut metrics.

    \begin{proof}
        Since $d$ is a cut metric with cuts $T_1, \dots, T_k$, $\exists \lambda_1, \lambda_2, \dots, \lambda_k > 0$ s.t. $d(X,Y) = \sum_{i=1}^k(\lambda_i \cdot d_{T_i}(X,Y))$, $\forall X,Y \in V$.

        Then,
        $\phi(d) =
        \dfrac{\sum_{\{u,v\} \in E} d(u,v)}{\sum_{\{u,v\} \in \binom{V}{2}} d(u,v)} = 
        \dfrac{\sum_{\{u,v\} \in E} \sum_{i=1}^{k} (\lambda_i \cdot d_{T_i}(u,v))}{\sum_{\{u,v\} \in \binom{V}{2}} \sum_{i=1}^{k} (\lambda_i \cdot d_{T_i}(u,v))} = 
        \dfrac{\sum_{i=1}^{k} \sum_{\{u,v\} \in E} (\lambda_i \cdot d_{T_i}(u,v))}{\sum_{i=1}^{k} \sum_{\{u,v\} \in \binom{V}{2}} (\lambda_i \cdot d_{T_i}(u,v))}$

        We now claim that, if $a_i, b_i > 0$, $\forall i \in [k]$, then
        \[ \dfrac{\sum_{i=1}^k a_i}{\sum_{i=1}^k b_i} \geq \min_{i \in [k]} \dfrac{a_i}{b_i} \]

        Let $\rho = \min_{i \in [k]} \dfrac{a_i}{b_i}$.

        Then
        $\dfrac{\sum_{i=1}^k a_i}{\sum_{i=1}^k b_i} = 
        \dfrac{\sum_{i=1}^k b_i \frac{a_i}{b_i}}{\sum_{i=1}^k b_i} \geq
        \dfrac{\sum_{i=1}^k b_i \rho}{\sum_{i=1}^k b_i} =
        \rho$

        Thus the claim is proved.

        Now,
        \[ \phi(d) \geq \min_{i \in [k]} \dfrac{\sum_{\{u,v\} \in E} (\lambda_i \cdot d_{T_i}(u,v))}{\sum_{\{u,v\} \in \binom{V}{2}} (\lambda_i \cdot d_{T_i}(u,v))} = \min_{i \in [k]} \phi(d_{T_i}) \]
    \end{proof}

    \begin{corollary}
        \[ \min_{\substack{d\\ d \text{  a cut metric}}} \phi(d) = \min_{\substack{d_T\\ d_T \text{ an elem. cut metric}}} \phi(d_T) = \Psi^*  \]
    \end{corollary}

    \begin{proof}
        Follows from Lemma~\ref{lemma:metric2} and Lemma~\ref{lemma:metric3}.
    \end{proof}

    \begin{exercise}
        Prove that each cut metric $d$ is a metric.
    \end{exercise}

    \begin{definition}[Normed metric]
        Given $\underbar{X}, \underbar{Y} \in \mathbb{R}^d$,
        \[ l_p(\underbar{X}, \underbar{Y}) = \sqrt[p]{\sum_{t=1}^{d} \cardinality{X(t) - Y(t)}^p} \]
    \end{definition}

    \begin{theorem}
        $l_p$ is a metric over $\mathbb{R}^d$, $\forall d \geq 1$, $\forall p \geq 1$.
    \end{theorem}

    We define $l_\infty$ as
    \[ l_\infty(\underbar{X}, \underbar{Y}) = \max_{t \in [d]} \cardinality{X(t) - Y(t)} = \lim_{p \rightarrow \infty} l_p(\underbar{X}, \underbar{Y}) \]

    We are interested in some specific metrics.
    Consider $l_1$, called \textit{Manhattan distance}, which has some nice properties.
    It is, essentially, just a linear function with absolute values.
    \[ l_1(\underbar{X}, \underbar{Y}) = \cardinality{\underbar{X} - \underbar{Y}}_1 = \sum_{t=1}^{d} \cardinality{X(t) - Y(t)} \]

    \begin{lemma}
        If $d : V \times V \rightarrow \mathbb{R}$ is a cut metric over $k$ cuts, then $f : V \rightarrow \mathbb{R}^k$ s.t. $d(X,Y) = \cardinality{f(X) - f(Y)}_1$, $\forall X,Y \in V$.

        If $X \subseteq \mathbb{R}^d$ is finite, then the $l-1$-metric over $X$ can be represented by a cut metric over $k = (\cardinality{X}-1)d$ cuts.
    \end{lemma}